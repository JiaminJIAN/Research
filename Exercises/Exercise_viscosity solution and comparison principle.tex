%GCE of WPI
%by Jiamin JIAN

\documentclass[12pt,a4paper]{ctexart}
\usepackage{CJK}
\usepackage{lipsum}
\usepackage{amsmath}
\usepackage{geometry}
\usepackage{titlesec}
\usepackage{amssymb}
\usepackage{epsfig}
\usepackage{float}
\usepackage{graphicx}
\usepackage{tabularx}
\usepackage{longtable}
\usepackage{amstext}
\usepackage{blkarray}
\usepackage{amsfonts}
\usepackage{bbm}
\usepackage{listings}
\geometry{left=2.5cm,right=2.5cm,top=2.5cm,bottom=2.5cm}

\begin{document}


\begin{center}
\textbf{Viscosity solution and comparison principle}
\end{center}

\vspace{12pt}

$\textbf{Exercise 1:}$

Consider the ODE
\begin{equation} \label{ODE}
    \begin{cases}
   |u'(x)| -1 = 0, \hbox{ on } x \in (-1, 1) \\
   u(\pm 1) = 0
   \end{cases}
\end{equation}


(1) Is $v(x) = |x|-1$ a viscosity solution of (\ref{ODE})?

(2) Is $u(x) = 1 - |x|$ a viscosity solution of (\ref{ODE})?

(3) Can you prove comparison principle?

\vspace{8pt}

$\textbf{Solution:}$

(1) For $|u'(x)| - 1 = 0$, we denote 
\begin{equation*} 
    F(x, u, p, X) = |p| - 1.
\end{equation*}
By the definition of semi-jets, for $v(x) = |x| - 1$, when $x = 0$, we have
\begin{equation*}
    J^{2, -} v(0) = ((-1, 1) \times \mathbb{R}) \cup (\{1 \} \times (- \infty, 0]) \cup (\{-1\} \times (- \infty, 0] ).
\end{equation*}
Thus for $x = 0$, there exists $(p, X) \in ((-1, 1) \times \mathbb{R}) \subset J^{2, -}v(0)$ such that
\begin{equation*} 
    F(x, v, p, X) = |p| - 1 < 0,
\end{equation*}
so, we know that $v(x) = |x| - 1$ is not a viscosity supersolution of $|u'(x)| - 1 = 0$, then it is not a viscosity solution of (\ref{ODE}).

\vspace{4pt}

(2) Similarly, by the definition of semi-jets, for $u(x) = 1 - |x|$, for $x = 0$, we have
\begin{equation*}
    J^{2,+} u(0) = ((-1, 1) \times \mathbb{R}) \cup (\{-1 \} \times [0, + \infty)) \cup (\{1\} \times  [0, + \infty) )
\end{equation*}
and 
\begin{equation*}
    J^{2,-} u(0) = \emptyset.
\end{equation*}
When $x \in (0,1)$, we have
\begin{equation*}
    J^{2,+} u(x) = \{- 1\} \times [0, + \infty), \quad J^{2, -} u(x) = \{-1 \} \times (- \infty, 0],
\end{equation*}
and when $x \in (-1, 0)$, 
\begin{equation*}
    J^{2,+} u(x) = \{1\} \times [0, + \infty), \quad J^{2, -} u(x) = \{1 \} \times (- \infty, 0].
\end{equation*}
Hence we can conclude that for any $x \in (-1, 1)$ and $(p, X) \in J^{2,+} u(x)$,
\begin{equation*}
    F(x, u, p, X) = |p| - 1 \leq 0,
\end{equation*}
then $u(x) = 1 - |x|$ is a viscosity subsolution. And for any $(p, X) \in J^{2,-} u(x)$,
\begin{equation*}
    F(x, u, p, X) = |p| - 1 \geq 0,
\end{equation*}
then $u(x) = 1 - |x|$ is a viscosity supersolution. For $x = 1$ or $x = -1$, $u(x) = 1 - |x| = 0$, so we know that $u(x) = 1 - |x|$ is a viscosity solution of the ODE (\ref{ODE}).

\vspace{4pt}

(3) To get the comparison principle for the ODE (\ref{ODE}), if we denote $\Omega = (-1, 1)$, we need to show that: let $u \in USC(\bar \Omega)$ and let $v \in LSC(\bar \Omega)$ be a viscosity subsolution and supersolution of (\ref{ODE}) respectively, if $u \leq v$ on $\partial \Omega$, then $u \leq v$ in $\bar \Omega$.

Suppose that 
\begin{equation}
    \max_{\bar \Omega} (u - v) (x) = \theta > 0,
\end{equation}
if we choose $\mu \in (0,1)$ such that 
\begin{equation*}
    (1 - \mu) \max_{\bar \Omega} u \leq \frac{\theta}{2},
\end{equation*}
then easily we can get
\begin{equation*}
    \max_{\bar \Omega} (\mu u - v) =: \tau \geq \frac{\theta}{2}.
\end{equation*}
For $\bar x \in \bar \Omega$ such that $(\mu u - v)(\bar x) = \tau$, we may suppose that $\bar x \in \Omega$. Otherwise $\bar x \in \partial \Omega$, if we further suppose that $\mu < 1$ is close to $1$ such that
\begin{equation*}
    - (1 - \mu) \min_{\partial \Omega} v \leq \frac{\theta}{4},
\end{equation*}
then as $u \leq v$ on $\partial \Omega$, we have
\begin{equation*}
    \frac{\theta}{2} \leq \tau = \mu u(\bar x) - v(\bar x) \leq \mu v(\bar x) - v(\bar x) = (\mu - 1) v(\bar x) \leq \frac{\theta}{4},
\end{equation*}
which is a contradiction. 

Consider the mapping $\Phi_{\epsilon}: \bar \Omega \times \bar \Omega \to \mathbb{R}$ defined by
\begin{equation}
    \Phi_{\epsilon} (x, y) = \mu u(x) - v(y) - \frac{|x - y|^{2}}{2 \epsilon}.
\end{equation}
Choose $(x_\epsilon, y_\epsilon) \in (\bar \Omega, \bar \Omega)$ such that
\begin{equation*}
    \max_{x, y \in \bar \Omega} \Phi_{\epsilon} (x, y) = \Phi_{\epsilon} (x_\epsilon, y_\epsilon),
\end{equation*}
then 
\begin{equation*}
    \Phi_{\epsilon} (x_\epsilon, y_\epsilon) \geq \sup_{x \in \bar \Omega} \Phi_{\epsilon} (x, x) = \sup_{x \in \bar \Omega} (\mu u - v)(x) = \tau \geq \frac{\theta}{2}. 
\end{equation*}
We suppose that $\lim_{\epsilon \to 0} (x_\epsilon, y_\epsilon) = (\hat x, \hat y)$ for some $(\hat x, \hat y) \in (\bar \Omega, \bar \Omega)$. Also, we have that
\begin{equation*}
    \frac{|x_\epsilon - y_\epsilon|^{2}}{2 \epsilon} \leq \mu u(x_\epsilon) - v(y_\epsilon) - \tau \leq M_{\mu} = \mu \max_{\bar \Omega} u - \min_{\bar \Omega} v,
\end{equation*}
then $|x_\epsilon - y_\epsilon|^{2} \leq 2 \epsilon M_{\mu} $. By sending $\epsilon \to 0$, we have $\hat x = \hat y$. Hence, the above inequality implies that
\begin{equation*}
    \mu u(\hat x) - v(\hat x) = \tau,
\end{equation*}
which yield $\hat x \in \Omega$ because of the choice of $\mu$. Thus we see that $(x_{\epsilon}, y_\epsilon) \in \Omega \times \Omega$ for some small $\epsilon > 0$. Moreover, we have
\begin{align*}
    0 \leq \liminf_{\epsilon \to 0} \frac{|x_\epsilon - y_\epsilon|^{2}}{2 \epsilon} & \leq \limsup_{\epsilon \to 0} \frac{|x_\epsilon - y_\epsilon|^{2}}{2 \epsilon} \\
    & \leq  \limsup_{\epsilon \to 0} (\mu u(x_{\epsilon}) - v(y_\epsilon)) - \tau \\
    & \leq (\mu u - v )(\hat x)  \leq 0,
\end{align*}
which implies
\begin{equation*}
    \lim_{\epsilon \to 0} \frac{|x_\epsilon - y_\epsilon|^{2}}{2 \epsilon} = 0.
\end{equation*}
Taking 
\begin{equation*}
    \phi(x) = \frac{1}{\mu} \Big{(} v(y_\epsilon) + \frac{|x - y_\epsilon|^{2}}{2 \epsilon} \Big{)},
\end{equation*}
we see that $u - \phi$ attains its maximum at $x_{\epsilon} \in \Omega$. By the definition of viscosity subsolution, we have
\begin{equation*}
    \frac{|x_\epsilon - y_\epsilon|}{\mu \epsilon} \leq 1,
\end{equation*}
which yields $\frac{|x_\epsilon - y_\epsilon|}{\epsilon} \leq \mu$. On the other hand, taking
\begin{equation*}
    \psi (y) = \mu u(x_\epsilon) - \frac{|y - x_\epsilon|^{2}}{2 \epsilon},
\end{equation*}
we see that $v - \psi$ attains its minimum at $y_{\epsilon} \in \Omega$. Similarly, by the definition of viscosity supersolution, we have
\begin{equation*}
    \frac{|x_\epsilon - y_\epsilon|}{\epsilon} \geq 1.
\end{equation*}
Then we can get
\begin{equation*}
    1 \leq \frac{|x_\epsilon - y_\epsilon|}{\epsilon} \leq \mu,
\end{equation*}
which contradicts with $\mu \in (0,1)$.

\vspace{8pt}

\textbf{Theorem:}

Consider the following PDE:
\begin{equation}
    H(x, Du) - f(x) = 0, \qquad x \in \Omega
\end{equation}
where $H: \Omega \times \mathbb{R}^{n} \to \mathbb{R}$. We suppose that
\begin{itemize}
    \item there is a continuous function $\omega_{H}: [0, \infty) \to [0, \infty)$ such that $\omega_{H}(0) = 0$ and 
    \begin{equation*}
        |H(x, p) - H(y, p)| \leq \omega_{H}(|x - y|(1 + |p|))
    \end{equation*}
    for $x, y \in \Omega$ and $p \in \mathbb{R}^{n}$,
    \item $H$ has homogeneous degrees $\alpha > 0$ with respect to the second variable, i.e. there is $\alpha > 0$ such that
    \begin{equation*}
        H(x, \mu p) = \mu^{\alpha} H(x, p)
    \end{equation*}
    for $x \in \Omega, p \in \mathbb{R}^{n}$ and $\mu > 0$,
    \item there is a $\sigma > 0$ such that
    \begin{equation*}
        \min_{x \in \bar \Omega} f(x) = \sigma > 0.
    \end{equation*}
\end{itemize}
Let $u \in USC(\bar \Omega)$ and let $v \in LSC(\bar \Omega)$ be a viscosity subsolution and supersolution of (4) respectively, if $u \leq v$ on $\partial \Omega$, then $u \leq v$ in $\bar \Omega$.

\end{document}
